\documentclass[../NenokuniMain]{subfiles}

\setcounter{chapter}{0}

\begin{document}
いつになっても、踏切の不安症な音色は変わらない。
私の子供の頃から、いや、そのはるかに遠い過去から、人は歪んだこの音を『危険』と結びつけてきた。
弾ける肉片へのスタートライン、つまり死に限りなく近い境界線。
今はゴールラインも兼ねる黄色と黒のストライプは、私の敗北条件\ruby{∧}{かつ}勝利条件。
眼の前を走る男を捕まえても、先へ飛び出しても、どのみち相手が死ぬことに変わりはない。
夜の\ruby{天照雨}{そばえ}。星屑の落下。
月明かりに照らされて、私の追走劇は幕を下ろす。

カムイ ヤヤヒサシク/

本名\ruby{鴨居良久}{かもいよしひさ}\scalebox{2}[1]{―}\ruby{Case13}{サッショブン}該当者。

 \\

遮断機の降りた踏切へ走り込んで、

カムイヤヤヒサシク/鴨居良久は轢死した。\\

\scalebox{2}[1]{―}終了。\\

けれど、私はここで止まらない。

真実はすぐそこに。

忘れ去られていく人々の、細切れになった世界を縫っていって。

一人ぼっちだと拗ねる彼女の頬を、思いっきり叩かないと。\\

PLAYBACK Label:Kamui\_YAYAHiSaShiKu\\
          TimeSet\\
            <@>\\
         2120/2/03\\
     19:28:21(UTC+0900)\\

好き……♥\\

小さい手で私の拳を握りしめる彼女。
凍える夜の冷たさを残したこの手は、いつにもましてぬくもりを感じる。
照れくささに、やり場のない目を電車の外に向けると――そこにいた。
カムイヤヤヒサシク。ホームの上。私の目の前に、窓を挟んでそこに立つ。
「間もなく、三番乗り場より――」アナウンスは私の乗る列車について。
時間はない。考える必要もない。脊髄反射的に行動する。\\
                開/閉\\
                開/■\\
「――駆け込み乗車はご遠慮ください」\\
到着列車からの人混みに紛れて、カムイは流れていく。
焦る必要はない。目を離すな。瞬きは禁止。
階段を下って改札を飛び越えて、私は走る。
そして――。

Seek 00:00

「あなただったのね」

カムイヤヤヒサシク、いや、こんなうざったい名前はもう忘れよう。

「そうね。そのとおりかも」

\section{}

「あなただったのね\scalebox{2}[1]{―}」\\
私はわたしの記憶を参照する。符号化された画像記憶は、
精緻に視覚と重ねられ、私の言葉を証明する。\\
「そう。覚えてくれてたんだ」\\
当たり前のことを彼女は言う。まるでそれが異常であるかのように。
やはり、彼女はこの世界の異端者。解け合えない異物。\\
「\scalebox{2}[1]{―}\ruby{蒼祇}{アオカミ}ツグみ、じゃなくてツグネ監理官」\\
彼女はしかし、その断絶を越えようとしていた。
私の名前を『うろ覚え』たくて、わざといい間違えようとしたり。

またあの音だ。
2分23秒前の振動と、同じ周波数、振幅。
サンプリング定理に保証され、私の記憶領域にバッファされる。

が、Eventが起こる。絶えず圧縮される私の記憶の、無視できない差異のこと。



傾きかけた午後の日に押されて、空から雨粒たちが降下してくる。
雲の裂け目。大きな口の喉奥から指す日差し。だったら雨は涎か何かかな。
晴れてるのにどうして。
傘の裏側を見上げてみれば、キラキラとスパンコールみたいに飾られている。

俗に言う狐の嫁入り、日照り雨、格好つけて\ruby{天照雨}{そばえ}と言おう。

校門前への坂を下る放課後、ふと思う。
私と同じように空を臨むあの人々は、私のように感じるのだろうか。

思い出。時間がかける、手間ひまかけた研磨は、記憶の経年劣化という概念と一緒に死滅して、
今の時代の思い出とは、忘却によってろ過された、正真正銘の忘れてはいけない記憶と同義語だ。
完全に、証書がついて、自動的で
\scalebox{2}[1]{―}その他諸々ややこしいことを全部一緒くたにすれば\scalebox{2}[1]{―}
ぶり返すことのない絶対的な記憶の放棄、その逆説をとって、絶対に忘れることのないと誓う記憶、
それだけの価値があるものしか残らない。

そして今日の雨の輝きは、それに比べればきっと何倍にも劣るものなのだ。

坂を下りきって数歩、私は目を閉じる。
あんなにも綺麗な空間。景色は簡単に想起できて、私のに精神的な潤いを与えてくれるはず。
だって、こんなこと滅多にないから。宝物を拾ったも同然。

だから高をくくっていた。

私だけのものなのだと。

さて、瞼を閉じれば脳裡  に雫の輝きが\scalebox{2}[1]{―}全くの灰色で塗り固められていた。\\

{\LARGE \hspace{50pt}\ding{118}}\\

「アタシがみんなと共感できるようになるために、アネクドートは必要だった。
灰色に止まった世界の中で、記憶に留められたすべてを書き起こしても、そこには何一つ心を動かされる物がなかった。
確かにその時、私の心は振動していたのに」\\
暗闇に響く声は、聞き覚えのある声だった。
関係ないとあしらえば、それで済む希薄な接点しか持たない人間。
そう思っていたのに。
路地裏の分岐路、入り組んだ街の隅に、光さえ歪ませる『重さ』がそこにある。
自らは空虚だと謙遜するそれに、私は共感を覚えていたはず。
全く違う印象に、私は認識の更新を余儀なくされる。\\
「あなたのことを私は知っている。もっと早くに気付くべきだった」\\
私は牽制した。お前のことはわかっている。情報を握るものが、この世界の勝者だ。\\
「そう、キミだから気付けた」\\
「私だから?」\\
「キミはアタシと同類だよ。魂の共振周波数が近いんだ。だから同じことに敏感になる」\\
「同じこと?私はあなたとは違う」\\
そう主張しなければならない。絶対にだ。眼の前の人間と自らの同一性を認めてはいけない。
私はこの社会を守るべき存在。ましてや相手はこの社会に復讐したがる存在。
似てはならない。私は必死に思考する。共通点とはなにか。持てる記憶の倉庫をくまなく駆け巡って、
しかし答えは一向に出てこない。

「そばえ。狐の嫁入り。晴天に垂れる神様の涎。それを綺麗と振り返れないのを悔やむ人間なんて、
あなたと私ぐらいでしょ。ね?」\\
図星だろう、論破したと見下される私は、何もかもをたった今失った。
殻になった私の頭の中、もっと言えば単色に染まっていた思い出に、さっきまで見ていた幻想が尽く剥がれて、
コンクリートみたいな壁が、過去と今、私を私と意味づける全ての前に立ちはだかっている。\\
「だからさ、私と一緒に行こうよ」\\
ヒビが入るぐらいには強い言葉。けれど、穏やかで幽かな語気。\\
「忘却の川に私達が浚われて、ネノクニに落ちるその前に」

% \chapter{はじめての悪運}
\Entry{楽園の庭師}{the first ill luck}
\section{Case12.}
青模様の丼は、どこか空腹に接続されている。
夜、夕食を取るためにカウンター席に着く男は、ごく平凡な会社員だ。
三日ひげ剃りをさぼった顔を掻きながら、ラミネートされた脂で粘つくメニューを手にする。
ありふれた日常、少なくとも、彼を脅かすものは何一つないこの空間に、
たった一つの異変が起こってからはもう、彼の逃避する場所は何一つ残されてはいなかった。

店員を呼び、注文を済ませ、角のテレビを慰めにみる。
シワひとつない紺のスーツを着たキャスターが、昨今の社会情勢について意見を求めていた。
『忘却権の今とこれからについて』そんなご大層な議題を上げておいて、
話の内容は幼稚園児でもまだマシな、
上っ面だけの\scalebox{2}[1]{―}これはあくまでも、男の感想だ。\scalebox{2}[1]{―}話ばかり。

つまんね事ばっかだな。

ひとり毒づくその時、目の前の画面は何処やら。

正確には、人に隠れていたのだ。\\
「となり、座ってもいいですか」\\
女性だった。しかも、美人だ。
髪はショートボブ。目つきは若干鋭いが、それが彼の好みにぴったりだった。
クールな大人の女。背も高く、胸もデカイ。それに挑発的な蒼い唇……。
決して完璧ではないが、はっきりとした現実感がそそる。
本能の羅列は瞬時に、その余韻は長く。
ほんの少しの幸運を逃しまいと、男は彼女をまじまじと見つめ、匂いも堪能して、やっと返事をした。\\
「お、おう、別にいいよ」\\
「そうですか。ありがとうございます」\\
やけに笑顔だ。男は怪しもうとしたが、その間を与えられることもなく、彼女は先手を打った。\\
「じゃあついでに、本人確認してもらってもいいですか」\\
\scalebox{2}[1]{―}本人確認。一瞬時間が止まった。
わけががわからない。単語の意味を正しく処理できず、彼の脳みそは一時的にスタックする。\\
「本人確認って?俺がなんかしたのかよ」\\
「まあまあ、そんなことはどうでもいいことですから」\\
ほら、と女性は一枚の写真と質問を渡す。\\
「この写真の人物、大鳥大介さん、あなたで間違いありませんよね」\\
人差し指と親指で、まるで汚い何かをつまんでいるようにして、彼の目の前に差し出された写真には、
たしかに、解答は一つしか持たない人間、つまり、男本人の姿が写っていた。

予定調和の返答を、悠長に待つ必要はない。\\
「そ、そうだけど……」\\
男は席から立ち上がる余裕もないようだ。未だに、自分の名前を赤の他人に尋ねられるこの不可思議な
現実に、まがりなりにも己の解答を模索していた。けれど制限時間は思いの外、短かった。

屈託のない、極端に口角の上がった笑みに、男はいまさら逃亡という選択肢を思い出す。\\
「それじゃあ良かった。\ruby{Case11}{忘却代執行}\scalebox{2}[1]{―}執行です」\\

\section{雨の中で}
忘れたくなるほど雨の多く降る街。
参照すれば私の特別な日々は常々、うるさい雨音で台無しにされていた。

待ち時間は既に一時間を超えて、目の前の車の通りも少なくなってきた。
雨足は強く、どうしてこうも私を虐めるのだろうか。
スカートに斑のシミ。これ以上玄関前で待っている理由を見つけられない以上、
庁舎の中に入ってしまってもいいのではないかという、甘い誘惑が私の思考を支配し始める。
広坂合同庁舎の三分の一を専有する、
『総務省情報局記憶監理室』は何を隠そう私が初めて働き始める職場だ。
増設された建屋は妙に出っ歯ていて、
私の職場そのものがこの地域行政の場から爪弾きにされているようにも見える。
その直感を支持するように、時々横切る職員の顔は、部外者を見る目のそれだ。
ただ、それも仕方のない事かもしれない。
青を基調とした支給レインコートは、この灰色の空気にはビビットに映える。
せっかく貰ったものだし、傘をさす手間を省けると羽織ってきたが、防御力はないに等しかった。
前が留められないのだ。明らかな欠陥品。不必要に目を引き、実用性もない。
正直言って今は邪魔だった。\\
{\tt Ev:Current\_Code\_is\_updated\\
\{新しい環境に負けないで、自信を持っていきましょう。\}\\
timeStamp:2132-05-12/09:22:43}\\
通知音とともに読み上げが始まる。
きっと私の感情が、一瞬でも基準値を超えて負の効果を持ち始めたからだろう。
限りなく人間に近く、しかし所々に不正解を持つ合成音声は、
わざとらしい音の外し方で、けれど流暢に\ruby{行動規範}{コード}を音読する。
もちろん欠陥は人為的で、開発者曰く多少のバカさ加減がなければ好かれないだろうという、
安易な対機械関係観に則った意向らしい。
たしかそういう旨のインタビューが、ニュースに書いてあった。
必要のないことをする。お節介とはこのことを言うんだろう。
なぜならどう考えても声色に合っていない。想像できる外見こそが、声の形だと思う私にとって、
この喋り方はまったくもってナンセンスなもの。バッサリと物事を割り切れるビターな大人の女性が、
こんなトンチンカンで不自然な喋りをするだろうか。
絶対にない。もっと気怠げに、そつなく言ってもらいたい。
幸い、音声にはフィードバックを与える事ができる。
私が理想とする声を想像し、明確にすればするほどそれに近づいていくという仕様になっている。
育てる、といえば聞こえがいいが、言い換えれば学習コストはお高く付き、年単位もするということ。
私がそれを意識し始めたのは、大学を出てからのことだから、まだ二ヶ月ほどしか立っていない。
やっとイントネーションを把握した外人のような感じで、まだまだ到着点には遠い。

ほら、あなたのお節介で、肝心の中身に関心が行かないじゃないか。
全く面識のない、おそらくはこれからも知り合うことのない人間につく悪態というものは、
どこか恥ずかしさもあるけれど、それでいて心地いい。が、やはり子供のやることだ。
こういったことは綺麗サッパリ忘れて、私にはやることがあるんだから。

「そのコート着てるんだ。ダサいのに」
あっさりと言われて、私は完全敗北。\\

\section{前日:面接}
202会議室は空室。

「それでは始めましょうか」\\
三人の面接官、その平坦の真ん中が気さくにそういった。
きっと受験者を落ち着かせるための所作だろうが、その目論見は端から失敗している。\\
「よろしくお願いします」\\
古臭いマナーなどない現代、堂々と椅子に座ることが許される。
彼女も普段通りに腰を掛けながら、当然の挨拶を返す。
ただ彼女が困ったことは、その言葉を誰に向けているのか自分でもわからなかったことだった。

203会議室も空室。

うす暗い廊下は節電対策。遠い突き当りの窓が唯一の日差しを恵んでくれる。
並ぶ会議室、ガラスの箱とはいえ距離がありすぎる。あてにはできない。

彼女の前、面接官三人衆は奇跡的にも完全に同一な\ruby{面}{ツラ}をしていた。
彼女は最初単なる三つ子、それもDNAの寸分違わぬ多胎児かと考えたが、いや自然の摂理を考えれば、
完全に同じということなどないのだと、すぐにアプローチを変更した。

申し訳程度の緑化、手入れの届かないプランターには三つ編みのパキラが収まっていた。
204会議室も空室。

三つ子ははっきり言って別人だ。皆異なる家庭環境で育ち、異なる価値観を持つ、
独立した人間にほかならない。ではなぜ顔が一緒なのか。
答えは単純だ。それが『推奨』であるから。
総務省の採用試験面接時推奨\ruby{顔面用光学扮装}{facing}は、平均的な成人男性の寸法\scalebox{2}[1]{―}
全頭高24cm・頭幅16cm・頭長19cm・頭囲60cm(あくまでも参考程度に)\scalebox{2}[1]{―}
をベースに、いかにも人の良さそうな\scalebox{2}[1]{―}よく笑う人間に特有な\scalebox{2}[1]{―}頬の膨らみ、
どんぐり目、額を隠さない清潔な短髪、その他諸々を添加した、不快感の極限まで薄められた男の顔なのだ。
もちろん声門も偽装して、三人は同じ声を出す。緊張を解す、いい意味で力のない声。
けれど芯は通っていて聞き逃しを許さない。\\
「では、手短にいきましょうか」\\
「はい」\\
彼女は小さく首を縦に動かした。

赤い点線と青い実線の走査するカーペットには、清廉潔白な省庁を象徴するように微塵の埃もない。
だが部外者のせいだろうか、紙くずが二三落ちている。これはきっとズボンに紙を入れたまま洗濯してしまったものだ。
特有の解れた繊維が顔を見せている。これは看過できない。
潔癖症の番人たる清掃ドローンはこの不届き者を容赦なく自らの胃袋、
二度と元には戻れぬ虚空に吸い込み、保たれるべき清廉な廊下を取り戻した。

205会議室は使用中。『会議中につき静かに!』

「では……情報局については」\\
近況の報告や官僚への志望理由、やる気本気度元気度上々。
評価はごく普通に高く、一般的に面接に最適化された人間であるという一点のみ。
ただ反応の異質な話題が一つ。\\
「記憶監査室執行課を希望する理由は」\\
「誰も志願しないからです」\\
「あぁ\scalebox{2}[1]{―}まあ、人気のない部署だからねえ。あそこは辛気臭いし。亡霊が住み着いてる」\\
左の面接官がため息混じりに言った。
彼は相当この部署で苦労させられているらしい。\\
「でも、誰かがやらないと」\\
「確かに。人手不足は現実だし、この国にとってはなくてはならない大切な部署だ」\\
「君が希望する理由もなんとなくわかるけれど」\\
「君の適正考査の点数、\ruby{信用記憶評価}{CMA}のポイントどちらを考慮しても、
君はもっと良いキャリアを歩めるはずなんだよね」\\
「それはわかっています。でも、私と\ruby{満}{おなじ}点をとった人間はたくさんいます。
それにみんな、上を目指す」\\
彼女の考え、ありきたりな回答に潜む異常さを、面接官は逃さない。
きっとこれはどうにも出来ない人間の類だ。三人の見解は一致した。\\
「だったら、私一人ぐらい上に行かなくてもいいかなって」\\
それが独りよがりのニヒリズムか、半端な人間がしばし罹患する『他人と違う病』の症状なのか。
あるいは正真の、なんら不純物のないただの理由と認めるべきなのか。
面接官三人は、彼女に聞こえない声で討論を始める。
端から見れば無言のまま座り込んでいる集団、目を開けたまま居眠りでも始めたのかと疑ってしまう
状況だが、漂う緊張がそれを強く否定して、ご託宣を遥か天上の神より賜る巫女のような荘厳さを彼らは演出していた。

207から210会議室は空室。211・212会議室は使用中。
最後の213号室、面接に使用中。ガラス張りの結界、その外周を健気に清掃ドローンは沿って、
決して立ち入ることの出来ない向こう側のゴミクズに執着している。

「わかりました。面接は以上です」\\
「ありがとうございました。失礼します」\\
女性は素直に立ち去った。
彼女の後、まだ温い椅子を見ながら、今度は誰にでも聞こえる声で三人は最終決定を下す。\\
「彼女の希望することですから。仕方のないことでしょう」\\
「Letheシステムの指標は」\\
「まあまあですね」\\
「人材として魅力的なんだがなあ」\\
「だから仕方ないことでしょ」\\
「わかった。上に送っておくよ」\\

終いになった面接の後、空になった会議室にやっとのこさで清掃ドローンは入室を許可され、
その床に散らばる埃や紙くず、消しカスを回収し始めた。
この清掃ドローンの長所は、決して床専用ではないところにある。
なんと平たく足首ほどもない身長を伸ばして、机の上まで掃除する事ができるのだ。

新たな分野を切り拓いた画期性の申し子は、しかし少々頭足らずだった。

意味あるゴミ。つまりプリントアウトされたコピー用紙までも、見境なく持ち去ってしまう。
その犠牲になった紙束の一番上。

写されている経歴・顔写真・氏名は全て、
新年度より新たに『総務省情報局記憶監査室執行課』
へと配属される\ruby{蒼祇}{アオカミ}ツグネ監査官のものだった。

\section{}
拉げた古い声が聞こえる。\\
「冤罪?古い概念だな。
概念とは死んだ知識だ。遠い昔の話。
俺がまだガキの頃だったらまあ、まだ虫の息ぐらいはあったかもしれんが」\\
暗い部屋。埃をちらつかせる電灯の光は、半径の狭い強力な閃光で、
無慈悲にも容疑者〈座る男
({\tt \ruby{R}{Right to be}\ruby{F}{ Forgotten}\ruby{I}{ Infringement}\textbackslash \ruby{S}{Subject}})〉
を照らし、その顔を暴き立てている。\\
「なあ、世間様っていうのは、ちと勘違いしてると思わんか?」\\
座る容疑者の耳元で、年配の刑事らしき人間〈尋問する男〉はニヤニヤしながら話しかける。
縮れた彼の長髪は俯く男の肩をかすめて、見ているだけでうざったい。
無精髭もそうだ。人と接するには少しばかり身なりが足りていない。\\
「記憶信用ってのはな、何も楽しい記憶を保証しますよってことじゃねえ。
まあ言葉のまんまだが、記憶を信用するってことだ。
つまり、アンタがどんだけだんまり決め込んでも、アンタの頭の中の記憶、
これさえあれば、全部立証できるってことなんだよ」\\
「証拠は無いだろ。証拠は」\\
男の話を挑発だと警戒したのか、座る男は持てる最大の抵抗を試みる。
が、すでに状況は変わり難い。そもそも、男の切り札は端から不発弾だった。\\
「だから言っただろ?証拠なんていらない。
どんな人間だってやったやってないかぐらいの記憶はある。
酔って覚えてないのだの、心神喪失なんざあり得ない。
それはどうしてかって?みんな頭の中にマシンを入れてるからだよ。
ほらここ」\\
〈尋問する男〉は〈座る男〉のこめかみを叩く。\\
「たとえ脳みそが寝過ごしても、機械が記録を保証してくれる。
記憶だけがものを言う世界。それがアンタの生まれたこの街であり国さ。観念するこったな」\\
「認めない。認めないぞ俺は」\\
これからの恐怖に歯ぎしりしながら、〈座る男〉は腹の底から小声で唸る。
まだ暴れまわれば、チャンスはあったかもしれない。
だがそうせずにちまちまと小言を吐き散らすだけの男など、たかがしれている。
彼の犯した罪もまた、お似合いのものだ。\\
「執行は明日だ。検査室が空いてないんでな。それからはメンタルヘルスに通ってもらうぜ。
最低でも一ヶ月。頑張って治せってことだよ。
隣人の盗撮動画で『致す』ってのは不健全だし、もっといいもんがあるだろ?」\\
彼は忘却権の侵害、つまり、他者の不都合な情報を保持或いは記憶したことによって隔離され、
執行されることになる。
執行とは忘却義務を遂行しない市民に対して、
総務省情報局記憶監査室執行課が遂行を代行することであり、また原義には含まれないが、
社会復帰を目的とした更生メンタルヘルスケアを受診することである。
彼はこのあと数カ月は社会から隔離されることになる。
そこで浄化され、社会的に都合の良い情報媒体として更生し、
健やかな市民生活に復帰する。幸福な、他人に迷惑のかからない、自分だけの世界に帰ることができる
\scalebox{2}[1]{―}はずなのだが。

だが、男は頑なに主張を続ける。無意味な、しかし耳障りな楯突き。
焦点は彼の記憶、保持する動画についての問答に移り変わっていた。
参考:流出した隣家の人妻の性的な動画(プライバシー保護の為一部伏せ字){\tt RFI\textbackslash T::file\{mv-a344439-*****v\}}。\\
「俺はそんなもんで\ruby{■■■}{不適切な表現}ない!」\\
「いいや、やった!」\\
「やってない」\\
「おい、いい加減にしろ。さっきも言っただろ?お前がやったやらないを決めるんじゃねえ。
お前の記憶が確かにソイツを\ruby{■■■}{不適切な表現}にして\ruby{■■■}{不適切な表現}って
証明してるんだよ」\\
「いいや違う。俺はもっと若いほうが\scalebox{2}[1]{―}」

静止する二人。
暗闇は人形の家の壁を倒すように開け、向こう側の景色が顔を覗かせる。
暗い壁は今や硝子の障壁に変わり、全体を見ればさながらミニチュア模型のようだった。\\
「はい、ここまでですね」\\
女性が話を始める。\\
「ちょっとこれ以上やると、あのー、中学生の皆さんには、ちょっと早すぎる会話があるので……」\\
今までの光景はすべて、立体膜内の映像に過ぎない。
埃のちらつきも、男たちの押し問答も、
全て透過膜内の\ruby{光学的設計粒子}{ODP}、その反射が見せるまやかしなのだ。\\
「じゃあちょっと振り返ってみますと、まずさっきの『やったやってない』って言い合うってやつは、
現実にはありえません」\\
大きく腕を使ってバツ印を作る女性。
彼女は記憶監査室の広報課に所属する女性である。

現在は中学三年生向けのセミナー、今後彼らが生きていくことになる社会の仕組み、その一部を
説明している最中である。
会場は中学校の体育館。
昔からそうしてきたように、光り物を見やすくするために、
全体の照明は抑えられ、室内はワクワクする薄暗さになっている。
と同時に、程よい明かりは睡眠にも適しているから、誘惑に負ける人間も出てくる。

スクリーンは状況の解体を終えた後、比較にならないほど単純な描画、つまり、
単調でつまらないゴシック体の説明文を長々と垂れ流し始める。\\
「記憶信用っていう言葉、あったと思うんですけど、
これがどういうことかって言うのは\scalebox{2}[1]{―}」\\
記憶信用とは、現代社会、
特にこの日本国において最も重要視されるものが個人個人の記憶であるということである。
つまり、記憶はすでに個人が専有するものから、その一部を公共物として扱うということでもある。\\
「まあ、なんのこっちゃってなるんですけど、要は、簡単にですよ、ほんとに簡単に言うと、
自分に関係する嫌なこと、例えばさっきの映像の中で問題になってたのは、
他人のエッチな動画を持っていた、記憶していたって言うことですけど、これはダメですってことなんですね。
だって撮られた人は恥ずかしいし、そんな動画があるって事自体が、すごく嫌なことですよね。
この動画が残ってる限り、その人はまたいつ、それを思い出しちゃうかわからない。
だからその動画を、完全に消しちゃうんですね。で、その動画の中身を知ってる人の記憶も消しちゃう。
こうすると、動画を撮られた人は、動画を撮られたことも、それが流出したってことも全部忘れて、
まあ、少なくともそのことは心配しなくて言い訳ですよね。つまり幸せになれる」\\
広報課の制服は薄い青に染まっているが、この薄暗い中では灰色に近くなる。
遠くの方に座る生徒からすれば見づらいし、余計に話から興味がそれる。\\
「皆さんの中にももしかしたらいるかも知れないんですけど」\\
一部がざわつく。\\
「いやほんとにいるんですよ。中学生の相談件数も決して低いわけじゃなくて。
月平均、三件あるんですね。だから身近な問題なんですよ!これ」\\
表示される棒グラフは、なだらかに連なっていた。
頂点の均しはすぐにできる。平均三件は確かに多い件数かも知れないが、
中学生たちを怖気づかせるためには、いささかに力不足だ。\\
「一番の方法は、まず断るってことなんですけど、彼氏さんがどうしてもとか、
嫌われたくないとかってあると思うんですね。でも私から言わせてもらえば、
そんなことせがむ男なんて、もうバッサリ、捨ててください。
男の子の方もそうですね。
動画を撮られるってことは、簡単に消せなくなること、忘れられなくなるってことですからね。
そういうことを考えられない恋人なんて、恋人じゃないです。はい」\\
経験談なのか、それとも単なる帰納的な結論なのか、深い議論を始める物好きが数人。\\
「じゃあ、具体的に誰に相談すればいいのかって言うと、基本的には、
電話相談ですね。メールもありますし、チャットもありますから、そこに連絡してくれれば、
二十四時間、いつでも私達が対応しますので。ぜひ、お願いします」\\
明示されるアドレスやらURLは、すでに配布されたパンフレットにも記載されている。\\
「ちなみに、相談員の指名もできまして、もちろん私も可能です。
ええと、もしよろしければ私、益田と申しますので、ぜひ相談ください。
\scalebox{2}[1]{―}あとそれと、後ろにいるんですけど」\\
女性職員は群衆の後ろを指差して「あの人が、私より『偉い』監査官って人でして、
実際にこの街を、守ってくれている人ですね」\\
いきなりの指名に驚く女性。
不意にでるお辞儀は、お辞儀というには浅すぎる。
彼女が、監査官という役職についている人間らしい。
一応はそう見える、と生徒たちは納得した。
しかし役名の仰々しさに比べて、彼女本人の持つ雰囲気は単なる公務員に限りなく近い。
間違いではないのだが、なにか拍子抜けだ。
真面目で、適度に勤労で、適度に怠慢なごく普通の人間。
だが彼女こそが、この社会を真に守る盾、その一人なのだ。
と、いっても生徒たちにはどうでもいいこと。
興味があるといえば、そのルックスだけだろう。
彼女は『そこそこ』魅力的だった。まだ目の肥えていない男子諸君には尚更だろう。
アシンメトリーなショートボブ。左を伸ばしている。
滑らかな曲線を描くスーツのシルエットには、性的な興奮を抱く人間もいるだろう。
顔立ちも整って、いかにもな大人の女性だ。実年齢よりも暫し、高く見積もられることも多いはずだ。
だがただ一つ、欠点といえば、その落ち着きすぎな目だろう。
人を見る目と言うよりは、転がる石に躓かないようと注視する目だ。
そうか、あれは人の向こう側をずっと見ている目、に違いない。生徒の一人はその表現にたどり着いた。
多分、それが適切だろう。

とはいえすぐに、興味は広報の会話に戻っていく。
言葉は場を締めようとしていた。
やっと終われる期待にと安堵に、どよめきは静かに去っていく。

講習は約一時間で終わった。途中脱落者は二十名ほど。
質問者は、ゼロだった。\\

「呼び出し、聞いた?」\\
体育館の左入り口、学童クラブにつながる通路から声が聞こえる。
後片付けの手伝い、置き去りにされたパンフレットを数冊拾っていた途中に、ツグネは横を振り向く。
両開きの鉄扉の片方によりかかり、ショートボブの女性が腕を組みながら立っていた。
新人を見定める目は、好奇に染まっている。
気持ち上向きな吊目に篭もるけだるさは、彼女の性的な魅力を引き出し、
その魅力に抗える人間は少ないだろう。\\
「すみません。講座中だったのでミュートにしてました」\\
「ごめんなさい。私がそうしてってお願いしたの」\\
「別にいいですよ。どのみち、迎えには来ないといけなかったから」\\
彼女\scalebox{2}[1]{―}\ruby{諒上}{リヨウジヨウ}\ruby{冴那}{サエナ}はツグネに歩み寄る。\\
「退屈な新人研修はここまで。さあ、受け取って。これは今から蒼祇ツグネ監査官、あなたのものよ」\\
諒上の手には黒革で仕立てられた手帳を手渡される。\\
「昔の刑事手帳みたい\scalebox{2}[1]{―}ですね」\\
「まあね。かっこいいし」\\
ツグネは丁寧に手帳を確認する。
淡い、濃淡の無い薄青の下地と、彼女自身の顔写真。これは面接時に撮影されたものだろう。
機能的にはほぼ市販の手帳と何ら変わりはないが、表紙裏のシンボル、
そして所々の余白にあしらわれた、偽造防止目的の二次元コードがちらちらと目につく。
まるで魔除けだ、ツグネはそう思った。実際、この手帳を認可の無い人間が持とうとするものなら、
たちまち警告ウィンドウが開き、口やかましくその行為の不適切さを説教される。\\
「メモ帳的に使ってもらっていいから。案外役に立つときは立つのよ、それ。
面倒だけどペンの一本ぐらい買ってさ、まあ、知的な感じ、演出してね」\\
そういって彼女は胸ポケットから手帳を取り出し、チェックリストにバツを重ねる。
お使いする子供のようだ。知的とは言い難いが、古い人間には好まれそうでもあった。
事実、立ち寄った校長には、好意を持って受け入れられた。\\




\section{通常業務}

銀面の部屋。
蒸し暑い空間の中には、轟々と唸る数台のデスクトップパソコンと、
それぞれに繋がれたディスプレイが痛々しく光っている。
うち二台かにははじめから表示端末が接続されずに、単にNASとして運用されているようだ。
ヘビーユーザかどうかはまだ判断できないが、いずれにせよ、
アルミホイルの壁紙を選択している時点で、居住者は正常な人間ではない。
コーディネートならまだしも、この執念じみた虫食いの塞ぎようを見れば、
明らかに強迫性の何かしらを患っていることは明白だ。

それは部屋の外に待つ、二人の監査官の一致する見解だった。

はじめはアパートの一室全体をアルミホイルで覆う計画だったらしいが、
キッチンからベランダへの窓枠にかけて侵食しかかった銀色を最後に、
目論見は非現実的と却下されたらしい。
散乱した厚紙の芯や、コンビニ弁当の空箱、ゴミ袋は室内の男性が、
よほど不摂生な生活を実践していると教えてくれる。

そしてここ二三日は、完全に自室にこもりっきりになり、
遂に男は、一世一代の危機に直面することになったのだ。

「あのー、開けてくれませんかね?」\\
乱暴に叩かれる扉。
すでに玄関は突破され、男二人組はすぐ目の前、
薄いパイン材を隔ててそこに悪態をついている。
何が何でも出るものか、と言いたげな沈黙。\\
「田上さん、いい加減出てきませんか。貴方の気持ちもわかります」と落ち着いた声。
二人組の年配、監査官\ruby{談義所}{ダンギシヨ}\ruby{賢治}{ケンジ}は説得する。
彼は同情する方法を得意としているのか、知ったような口ぶりで室内の男に、
解錠を促す。\\
「私達は、あなたの為にここにいるんです」\\
だがその建前というか、前口上は男の気に触れてしまった。\\
「俺の為に?笑わせんじゃねえよ!あんたらが俺に何するかなんて、とっくにわかってんだよ!」\\
「ならなおさら出てきてください。あなたはまだCase12、保護処置対象だ。
この状況が悪化すればどうなるか、わかってますよね」\\
Case12とはなにか。一言で説明することは難しい、多大な行政的法的プロセスの末に導き出された
宣告であるが、簡潔に言えば『社会からの一時的な隔離、治療』だ。\\
「なあ、爺さん。こんなボイルになりたがってる野郎なんて、とっとと引きずりだしゃいいじゃねえか」\\
しびれを切らした片割れ、監査官\ruby{疋島}{ヒキシマ}\ruby{冬馬}{トウマ}は、談義所に
耳打ち\scalebox{2}[1]{―}しかし明らかに扉を貫通する音量で\scalebox{2}[1]{―}する。\\
「おい冬馬、乱暴なことはできんよ。曲がりなりにもまだCase12だからな。Case13じゃあ\scalebox{2}[1]{―}」\\
「\ruby{Case13}{殺処分}だって!」

\section{}

扇風機の強い風が、所々にはみ出た紙を揺らしている。
大きな棚、カラーボックスを繋ぎ合わせただけの、どう見てもお粗末な代物だが、
持ち主が望む機能を果たすには、十分な容積を確保していた。
それに敷き詰められたCD-ROMやBlue-ray Disc、記憶媒体としては数世代前の遺物たち。
積み重なった数々の間に挟まれている栞は、どうやら伝票のようなものらしい。
だが正規の小売業者や卸売業者のような、工業的な洗練さがなにもない、
ただの文字と数字、二次元コードの無秩序な集合体。
おそらくは個人営業にしばしありがちの、がさつな在庫管理の副産物だった。
それを証拠付けるように、弛いタンクトップの下着と、パンツしか履いていない小男が、
懸命に青く光るモニターを睨みつけていた。平日の昼間に。
三枚のディスプレイに囲まれて、だが活用できていない二枚には、
複数のSNSや掲示板、
読めない文字\scalebox{2}[1]{―}半角カタカナや不自然な空白(しかも全角)・記号の挿入されたものだ\scalebox{2}[1]{―}
の散乱した受信メール一覧表。
その中の一つ、真っ当な意味を持つそれを開いて、男はリンクをクリックした。

「……へへっ、こいつぁちょっと刺激的すぎるかもなぁ、なんて」\\
独り言、に思えるが、しっかりとした会話を続けている。
つまり画面の向こう側の誰かと通話しているということだ。\\
「まあ三日後ぐらいかねぇ。焼くのに時間かかりそうだし、最近調子悪くって」\\
使い古された座椅子の横には、古いデスクトップが並べられている。
おそらくは、それぞれが単一作業に特化して組み立てられたBTOマシン。
カスタマイズもされていて、通常より多くの光学ドライブを備えている。\\
「レーベルなしだったら、多少短く出来るっちゃあ出来るんだけど、
やっぱそこは妥協できんからねえ」\\
ペイントソフトには、
円盤状に加工された写真の上にロゴをデザインしたものが表示されていた。
市販のプリンタで印刷するつもりだろう。
男にはこだわりのあるデザインらしいが、どう見ても素人の練習課題、
お粗末なただの記号的な意味しかないレーベルだ。
ソフトの操作練習にはなるだろうが、これを渾身の作だと見せびらかされても、微妙な返事しかできない。
そんなものも妥協できず、納期をいたずらに伸ばす男に、通話相手は愛想を尽かしているようだった。
手短に話を済ませ、すぐに通話を切断した。

物理的な回転音が、一斉に響き始める。
並列に置かれたマシン達が、データの書き込みを開始した合図だ。
これを少し見守って、順調に作業が進んでいるのを確信してから、男はつかの間の休憩に入ろうとする。

おもむろに下半身を露出させ、ディレクトリを物色する。
自慰行為にふける男の顔は、やけに達観しているように見える。
だがその奥底には、醜悪な欲望や身勝手な充足感が渦巻いている。
この男は特にそうで、興奮の原因は酷く自己中心的かつ、非倫理的だ。

曰く、この醜態を収めた動画の共有は、自らと彼女の心理的つながりだという。
普段は涼しい顔をして日常を過ごす彼女の表面、その裏側にある卑猥な姿、
その隠された事実を知るのは、己と彼女のみ。
つまり、この世界で彼女と自分だけが、彼女の言わざる真実、
人々が欲してやまない好奇の対象を知っているという優越感。
人の秘密を、しかも本人の知らぬところで握るのは、想像し難い優位性を与えてくれる。
それが非社会的な順位だとしても、この男にはそれで十分なのだ。
町を出て歩く少年少女の痴態を流布できるという能力、情報を持つという全能感。
俺が外に出れば、こんな奴らの人生なんてあっという間に壊せるという自負。
だがそれを実行しないと豪語する、一方的すぎる心優しさと同情、共感。
それを胸に括って、想像の中の彼女を抱くことに、男は満足している。\\
「っち、ガキのくせに立派なモンぶら下げやがって」\\

だが僅かな卑下が、全てを台無しにすることは、往々にしてあることだ。

今回は、完全に自業自得ではあるが。

如何せん、やり過ぎだった。

「……十二時五十分、忘却権侵害で逮捕」

いつの間にか蹴破られていた自宅のドアに、全く気づかなかったのだから。

\section{}

総務省情報局記憶監査室執行課監査官、\ruby{蒼祇}{アオカミ}ツグネは炎天下の中を駆け抜けていた。

長い新人研修を終えて、やっとのことで配属されたというのに、初日からこの肉体労働。
汗だくの体が、彼女のこれからを占うように、不快な湿り気を常に与え続ける。

彼女が張り切って部屋を訪れたとき、そこには数人の人間が居たが、
彼女の指導係の二人は、皆出払っていた。
代わりに手渡されたのは、住所の書かれた紙ナプキン。

やっとの思いで到着した先には、不機嫌な顔をした女性が立っていた。

「遅い」\\
「すみません」\\
内心、事前に知らせてくれなかったことに毒付くが、それを表に出せばおしまいだ。
彼女は吐き出したくなる言葉を抑えて、代わりに上がった息を整えることにした。\\
「今日のこと教えてもらえなかったの?」\\
「はい」\\
正直に答える。\\
「はあ\scalebox{2}[1]{―}」長い溜息と「ごめん」\\
予期せぬ言葉に、彼女はごく自然に驚いてしまった。
理不尽なハラスメントでもあるのだろうと、勝手に予想していたが、彼女は違ったらしい。\\
「課長の不手際だから、気にしないで」\\
「あの、課長って言うのは」\\
彼女は、行き先の書かれた紙ナプキンをみて、一人の男性を思い出す。\\
「それ、\ruby{城至}{きし}課長のやつ。あのハゲジジイなんの知らん顔でこれ渡したんでしょ」\\
「え、あ、はい。ずいぶん優しい方だと」\\
「テキトウなだけよ。自分と、自分の食事以外にね」

彼女\scalebox{2}[1]{―}\ruby{諒上}{リヨウジヨウ}\ruby{冴那}{サエナ}は、ツグネの教育係の一人。
ショートカットとその体格が相まって、男性と見間違う人間も多いだろう。
顔も中性的で端正だが、
優しさとは無縁の、厳しい目つきを崩さない姿勢に、気圧される人間もまた多い。
スーツを着れば、引き締まった体、丁寧に鍛えられた長い四肢が、彼女のアイデンティティとして機能する。

「あ、自己紹介がまだか。私は諒上冴那。まあ先輩だけど、あんま硬くならないで。
そういうの気にしないから」\\
「よ、よろしくおねがいします。本日付で配属されました、蒼祇ツグネです」\\
「ツグネ、でいい?」\\
「はい。お願いしまうす」\\
「じゃあ私のことは適当に。私って分かればいいから」\\
彼女はあまり、自分のことに無頓着らしい。\\
「それじゃあ、まず最初のお仕事、やってみよっか」\\
そう言って突き出したのは、小柄で無精髭を伸ばしっぱなしにした男だった。\\
「お仕事、ですか」\\
「そう、まずはコイツの記憶消去から」\\
「わかりました」\\
「案外抵抗ないのね」\\
感心するように顔を緩める。\\
「なにかおかしいですか?」\\
「別に。ただみんな最初は嫌がるんだよね。あんたもそうならないといいけれど」\\
まあとりあえず、と諒上はツグネと男を連れて(引きずって)路肩に停まった車に向かった。

約一ヶ月の間、彼女たち『執行課』は違法ポルノ動画関連のネットワークを虱潰しに摘発し、
大方の供給源を壊滅してきた。

認可の下りていない、俗に言う裏動画の大半は、
年端のいかない少年少女たちのホームメイドばかり。
これがなぜ赤の他人に渡るのかといえば、つまり、リベンジポルノというものだった。
インターネット上にアップロードされた動画は、
すぐにネットワーク上を巡回するAIによって削除されるが、やはり物理的な遅延はどうしても発生してしまう。

その一瞬を掠めて、動画をローカルなストレージに貯め込むアーカイバーから、少年少女の
忘却権を保護することが、監査官の重要な仕事の一つだ。

特に、オフラインな記憶媒体に保存されたデータの押収や破壊、及び侵害者の記憶を消去するのが、
執行課の基本的な業務。
その性格のため、デスクワークよりも脚を使う仕事が多く、
時に警察との合同捜査も行う彼らは、中央省庁の中でも花形の部署の一つだ。
この社会の平和を、自ら守り抜くという自負と名誉。
当然、彼女もそれを目当てに、この部署への配属を希望した。
そして、適正考査と見事に一致し、彼女は自らのキャリアを積み上げていくことになる。
今日はその記念すべき第一歩、彼女の担当した事件の解決日。
だから、その緊張は凄まじいものなのだろう。

清潔感を漂わせる、ショートカットの黒髪を、頻りに弄るツグネを、
彼女の先輩であろう女性が、その行為を咎める。\\
「そんなに慌てたって、何も良いことないわよ」\\
低めの、落ち着き払った声。\\
「すみません、諒上さん」\\
「別に謝ることでもないけど、無駄な消耗には注意したほうが良い」\\
彼女\scalebox{2}[1]{―}\ruby{諒上}{リョウジョウ}\ruby{冴那}{サエナ}は、その豊富な経験から忠告する。\\
「

数分後。

突入の許可が下る。
首元や脇下に青い染みを作る多くの警察官達が、一斉に狭いアパートの一室になだれ込んでいく。
それに続いて彼らとは雰囲気の違う、黒いスーツの男性監査官が一人。
この強行の許される、揺るぎ難い礼状を盾に、小さなソドムの市へと踏み込んでいく。

「はい動かない」\\
オールバックの男性が、よく通る大声で住人の注意を引く。
賽の河原で積み上げられた石の塔のような、未開封のCD-ROMの山が続々と崩壊していく。
怖じ気ついて奥のリビングに引きこもるつもりか、ふすまを勢い良く閉じ、固い決意を見せつける。\\
「早く出てこいよ」\\
説得とは程遠い高圧的な命令。
ただ子どもがそうであるように、怒鳴るだけでは逆効果。
それをわきまえて、三人組の中でもっとも若い監査官が、その役割を買って出た。\\
「こんにちは、町田さん。情報局の者です」\\
騒然とした現場を中和する、落ち着き払った女性の声。優しすぎて逆に怪しく思えるほどの、慈悲に満ちた声色。\\
「開けてください。抵抗しても無駄ですよ。私達も手荒な真似はしたくないんです」\\
幼児に向き合う保育士の口調だ。
ただそれが癇に障ったのだろうか、部屋の主は殺気立った言葉を吐きつける。\\
「テメェら不法侵入だろうが! 俺を何だと思ってんだクソどもが!
さっさと出てけ! 出てけ! 死ねよ! くそったれが!」\\
「子どもかよ……」\\
呆れ返ったと、男は手狭なキッチンを物色する。
まだ脂がこびり付いたコンロや、水垢まみれのシンク、雑に積み上がったコンビニ弁当のトレイや
ペットボトルのほうが、彼にとってはまだ興味の持ちやすい物だった。
茹でたてのインスタント麺が、無造作にシンクの上で散乱している。
こりゃあ食べられんな、と顔をしかめる。

どうしますか、と警官の一人がまた奥に待機しているスーツの女に相談する。
静かな傍観者を気取っていたつもりだろうが、その仕草はすでに苛立ちで支配されている。

「やっちゃて」\\
彼女は判断した。
強行突破やむなし。
警官二人がふすまを蹴飛ばす。

「はあ? テメェら頭おかしいんじゃねえの」\\
予想外の行動に、みすぼらしい男が慌てふためく。\\
「はい逮捕して」\\
小柄な女性は、先程の声とは打って変わり堂々とした態度で、家主に宣告する。\\
「……十二時五十分、忘却権侵害で逮捕」\\
警官が手錠を嵌める。細い、手入れの怠った不健康な腕。\\
「はいじゃあツグネ、外」\\
奥から指示が聞こえる。
彼女\scalebox{2}[1]{―}\ruby{蒼祇}{アオカミ}ツグネ\scalebox{2}[1]{―}は、その折れそうな腕を
引っ張って、男を部屋の外に引きずり出した。\\
「離せ、クソ女! ブス!」\\
「うるさい」\\
「ああ、イタイイタイ痛い!」\\
ついカッとなって腕をつねる。
それでも抵抗は続けられて、男は脚を放棄した。
体重がどっとかかる。\\
「はあ、立って!」\\
持ち上げようとするが、成人男性一人を持ち上げられるほどの腕力は、ツグネに備わっていない。
だがすぐに助け舟が出る。
彼女の先輩、強面な女性が男の型を鷲掴みにして、
強引に、本当に引きずり出して、連れ去っていった。\\

男の行き先は、車だった。

強引な移送に尻を痛めたのか、腰を気遣う老人のように、自らの臀部を擦る。
骨ばかりの体には、そうとう衝撃が響いただろう。
特に鉄版の階段を降るときは、その音に下の階の住人も、流石にやり過ぎだろうと見つめていた。\\
「ツグネ、録画始めて」\\
「了解しました」\\
ツグネは助手席からカメラを回す。
後部座席に座る男とツグネの同僚、その二人の顔をのみ写すように。
「それじゃ、始めよっか」\\
「開始します。3、2,1」\\
カウントダウンの後、女は改まった声を整えて、一瞬に誠実な公務員へと様変わりする。
「2080年5月20日、12時57分執行開始。執行担当は、総務省情報局記憶監査室執行課、
監査官\ruby{諒上}{リョウジョウ}\ruby{冴那}{サエナ}\scalebox{2}[1]{―}」\\
諒上は顎をしゃくり、ツグネに催促する。\\
「同監査官、蒼祇ツグネ」\\
「それでは、\ruby{町田智則}{マチダトモノリ}さん、これから私のする質問には、
すべて誠実にお答えください。偽証した場合によっては、罰則が課せられますので、ご留意ください」\\
暗記している文言を、綺麗さっぱり言い終わった後に、諒上の口角は酷く曲がり、その悪魔的な純粋さをむき出しにする。
おもちゃを前にする子ども、あるいは、目の前の巨悪に興奮する正義の味方だ。\\
「町田さん、あなたは二十三名の忘却権を侵害しています。また、余罪を鑑みれば、おそらく被害者の人数は五十人を優に超える。
これらを認め、然るべき処置を受けることに、あなたは同意しますか」\\
「認めま\scalebox{2}[1]{―}」\\
おそらくこの先に続く語尾は、『ません』。だから諒上はそれを遮った。
そして集音マイクにはかろうじて取り上げることの出来ない音量で、町田の耳に言葉を流し込む。\\
「認めろ。いや、お前は認める。じゃないと\scalebox{2}[1]{―}」\\
諒上脅しに、町田も彼女と同じ小声で返す。
それが暗黙の前提、この空間を支配する掟であると、恐怖の内に服従するように。\\
「女が男のタマ触るもんじゃねえよ……」\\
震える唇に、汗を濡らす。\\
「でも好きなんだろう。ドマゾ野郎が。あたしらには全部わかってるんだよ」\\
「それ、プライバシーのしんが\scalebox{2}[1]{―}」\\
諒上は膝で男の股間を柔らかに小突く。小手調べ、強度の確認。
それだけでも十分に恐ろしい脅迫だ。少なくとも世の男性にとっては。\\
「あんたのタマ2つで、今もまだ夜も不安で満足に眠れない少女たちの対価なるんだったら、
いいよ、喜んで潰してあげる」\\
押し付ける力は更に強く。物理的な圧力と、心理的な圧力は線形だ。\\
「わかった。わかったから、潰さないで」\\
「じゃあ、認めます。……早く言え」\\
「\scalebox{2}[1]{―}認めます」\\
声を張る諒上。\\
「つまり町田さん、あなたは自身の執行に同意する、ということですね」\\
「\scalebox{2}[1]{―}はい」
「それでは実行します。ツグネ、やって」\\
緊迫したやり取りを、ただ傍らで静観していたツグネは、
半分意識が乖離していたが、すぐに自らを取り戻した。\\
「では町田さん、私の人差し指を見てください。意識を集中して。
それから左右に振りますので、目で追ってください」\\
ツグネは立てた右手の人差指を、町田の前で左右に運動させる。
右、左、右、左。前時代的なエセ催眠術のような。
ただ現代において、この行為はとてつもない影響力を持っているというのが、決定的な相違点。
これはいわば脱感作と再処理。
町田の、いや、この国に住むほぼ全ての人々の脳に根付いたマイクロマシンが、
宿主のシナプスを繋ぎ変えていく、不可逆的な忘却術。
指先の運動は補助的なものだが、その裏には、人類の英知が総動員されて、
男の穢れた記憶、この社会に存在してはいけない情報が浄化されていく。
この男の記憶はすでに、この男個人のものではなく、
つまり宝箱に仕舞っておく宝物と認められず、公にするべきもの、
大衆の目に晒すべく、広大なネットワークの海に流される、全国民共有の財産なのだ。
だから、保安する必要がある。誰だって、宝石に傷を付けられたくないだろうから。
そしてその役目こそが、ツグネ達『監査官』に与えられた、
この社会の守り手たるにふさわしい使命なのだ。

人々が自由に自らの記憶を手放すことが可能になって、早くも半世紀が経ち、良き市民のあり方とは、
悩みや鬱憤といった、ストレッサーを意識的に忘れ、清らかに生きることであり、
それを貯め込む人間は、一般に信用されない。
だからといって、犯罪や非道徳的な行為の記憶が見逃されるわけではない。
恣意的な記憶の忘却は全てシステムによって記録される。いわば忘却の履歴。
その個人をもっとも忠実に描く、唯一つの情報。これは公共物であり、
正当な理由さえあれば市役所の窓口で誰でも閲覧できる。
もっとも、詳細情報は限られた人間、情報局の監査官にのみ提供される。
監査官とは、人々の忘却活動を監査し、時に是正する官僚のことである。

忘却権、かつては忘れられる権利として生まれたこの若い人権は、しかし何よりも尊重される
権利として、この世界の頂点に座している。
これを侵害する人間から、権利者を保護することも、監査官の重要な役割である。

特にこの記憶信用社会において、人の恥や汚点をいつまでもこねくり回す人間というのは、
寛容と調和を第一とする社会秩序に反する不穏分子である。
記憶とはもはや個人の専有物ではなく、社会全体に共有される公共物であるという意識が普遍化した今、
それを非倫理的に活用したり、あるいはいたずらに思い出させる行為というものは、
到底許される行為ではない。
後ろ指をさされて生きることは、誰だって良しとしない。

『執行完了』と結果を示すログの数々が、監査官二人のウィンドウに表示される。
これは彼女たちの頭の中にのみ存在するディスプレイだ。
その他にも多くの情報が表示されるが、殆どは機密指定されたものばかり。
照会された個人情報、内容は氏名、年齢、性別、住所、管理番号
その他諸々の企業から提供された情報、そして個人が抹消してきた記憶の履歴だ。
記憶そのものを覗き込める監査官にとって、この表示はあくまで補佐的なものだが、
それでも、消去してきた記憶の履歴は、個人を計る上でもっとも参考になる情報の一つだ。
諒上の脅しは、町田の過去、拭い去った記憶のうちから導いたものだった。

「お疲れ様でした」\\
諒上は手っ取り早く事務的な作業を済ませ、早々に車から降りていった。
ツグネもそれに続き\scalebox{2}[1]{―}その前に大切なことを思い出した。
町田の今後についてだ。\\
「町田さん」\\
「はい!?」\\
「あなたには今後、社会復帰のためのセミナーを受講してもらいます。
その後、必要に応じて捜査に協力してもらうため、取り調べに応じてもらう可能性もあります。
自分に答えられるものは、はっきりと答えてください。それが、健全な社会への、然るべき奉仕です。
市民としての義務を果たしてくださいね」\\
それから彼女は足早に車を離れた。
そして残された男は、自らの内の喪失感に気付き、しかしすぐに忘れ、果たすべき義務を思い出す。\\
「それじゃあ行きましょうか」\\
知らぬまに入れ替わった運転手は、どこかへと男を運んでいった。

\end{document}